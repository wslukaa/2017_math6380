\documentclass[twoside]{article}
\usepackage{amsmath}
\usepackage{mathtools}
\usepackage{amsthm}
\usepackage{amssymb}
\usepackage{pb-diagram}

\usepackage{bbm}
\usepackage{color}

\usepackage{algorithm}  
\usepackage{algorithmic}  
% \usepackage{psfig}

\theoremstyle{definition}
\newtheorem{thm}{Theorem}[section]
\newtheorem{lem}[thm]{Lemma}
\newtheorem{prop}[thm]{Proposition}
\newtheorem{cor}[thm]{Corollary}
%\newenvironment{pf}{{\noindent\sc Proof. }}{\qed}
\newenvironment{map}{\[\begin{array}{cccc}} {\end{array}\]}
\newtheorem*{property}{Property}
\newcommand{\DS}{\displaystyle}

\theoremstyle{definition}
\newtheorem*{defn}{Definition}
\newtheorem*{exmp}{Example}
\newtheorem*{prob}{Problem}

\theoremstyle{remark}
\newtheorem*{rem}{Remark}
\newtheorem*{note}{Note}
\newtheorem*{exer}{Exercise}
\newtheorem*{example}{Example}

\setlength{\oddsidemargin}{0.25 in}
\setlength{\evensidemargin}{-0.25 in}
\setlength{\topmargin}{-0.6 in}
\setlength{\textwidth}{6.5 in}
\setlength{\textheight}{8.5 in}
\setlength{\headsep}{0.75 in}
\setlength{\parindent}{0.25 in}
\setlength{\parskip}{0.1 in}

\newcommand{\lecture}[4]{
   \pagestyle{myheadings}
   \thispagestyle{plain}
   \newpage
   \setcounter{page}{1}
   \setcounter{section}{0}
   \noindent
   \begin{center}
   \framebox{
      \vbox{\vspace{2mm}
    \hbox to 6.28in { {\bf Mathematical Introduction for Data Analysis \hfill #4} }
       \vspace{6mm}
       \hbox to 6.28in { {\Large \hfill #1  \hfill} }
       \vspace{6mm}
       \hbox to 6.28in { {\it Instructor: #2, HKUST \hfill Scribe: #3} }
      \vspace{2mm}}
   }
   \end{center}
   \markboth{#1}{#1}
   \vspace*{4mm}
}


%%%%%%%
% Some commonly used notations in Learning Theory
%%%%%%%

\def\R{{\mathbb R}}
\def\N{{\mathbb N}}
\def\L{{\mathcal L}}
\def\Z{{\mathbb Z}}
\def\X{{\mathcal X}}
\def\Y{{\mathcal Y}}
\def\E{{\mathbb E}}
\def\diag{{\rm diag}}
\def\sign{{\rm sign}}
\def\span{{\rm span}}
\def\supp{{\rm supp}}
\def\Pr{{\rm Prob}}
\def\H{{\cal{H}}}
\def\D{{\cal{D}}}
\def\U{{\cal{U}}}
\def\ie{{\it i.e. }}
\def\etc{{\it etc. }}
\def\etal{{\it et al. }}

\def\De{{\Delta}}
\def\P{{\rm Prob}}

\begin{document}

\lecture{Lecture 5. LDA, SIR, MDS and RKHS}{Yuan Yao}{Zhao Yuqi}{Mar, 13, 2017}
\section{Review of PCA}
In the previous lectures, we discussed Principal Component Analysis(PCA) and its generalitzion. 
\begin{enumerate}
\item Classical PCA in two different view:
\begin{enumerate}
  \item Geometry: Looking for best affine space to approximate a high dimensional Euclidean data. 
  \item Statistics: For a signal-noise model $Y = X+\epsilon$, recovering signal direction from principal component analysis on noisy measurements $Y$.
\end{enumerate}
\item Robust PCA: Looing for the decomposition $X = L + S$, 
      where \begin{itemize}
            \item $L$ is a low rank matrix;
            \item $S$ is a sparse matrix.
          \end{itemize}
\item Sparse PCA: Locating sparse principal component.
\end{enumerate}
Those algorithms are all efficient unsupervised dimension reducation methods. However, with label attaching data, some supervised feature extraction techniques can be used to increase the computational efficiency. 

\section{Linear discriminant analysis}
[Fisher's] \emph{Linear discriminant analysis}(LDA), like PCA, is look for linear combinations of features which best explain the data. However, LDA explicitly attempts to model the difference between the classes of data. 

For given data $(X_i,y_i)$, where $X_i \in \mathbb{R}^p$, and $y_i$ is discrite in ${\{1,2,...,K\}}$ not ordered.

In LDA, we want obtain feature vectors such that captures most variance between class and meanwhile discard the variance within class. 

Consider between class convariance matrix 
$$\hat{\Sigma}_B^{p\times p} = \frac{1}{K} \sum_{k=1}^K(\hat{\mu_k} - \hat{\mu})(\hat{\mu_k} - \hat{\mu})^T ;$$  
and within class covariance matrix 
$$\hat{\Sigma}_W^{p\times p}  = \frac{1}{N-K} \sum\limits_{k=1}^K\sum\limits_{y_i=k}(X_i - \hat{\mu}_k)(X_i - \hat{\mu}_k)^T ,$$
where $\hat{\mu}$ is sample mean and $\hat{\mu}_k$ is within class means, i.e. 
$$\hat{\mu}_k = \frac{1}{N_k} \sum\limits_{y_i=k} X_i.$$ 

Therefore $$ R(w) = \frac{w^T \hat{\Sigma}_B w}{ w^T \hat{\Sigma}_W w} $$ 
measures, in some sence, `signal-to-noise ratio' in terms of direction $w$. 

Intuitively, if $\hat{\Sigma}_W$ is inversiable, the eigenvector corresponding the largest eigenvalues of $\hat{\Sigma}_W^{-1} \hat{\Sigma}_B$ will maximize $S$. Accordingly, the best feature vectors would be eigenvectors corresponding top $k$ eigenvalues, i.e.
$$U_k = [u_1,u_2,...,u_k], \quad u_k \in \mathbb{R}^n, $$
where $\hat{\Sigma}_B u_k = \hat{\Sigma}_W \lambda_k u_k$ and $\lambda_1 \ge \lambda_2 \ge ... 
\ge \lambda_k$.


\begin{note}
 For \emph{Generalized Eigen Decomposition}(G.E.D) problem $\hat{\Sigma}_B \phi =  \lambda \hat{\Sigma}_W \phi $, it is more efficient to solve Eigen Decomposition problem $\hat{\Sigma}_W^{-\frac{1}{2}} \hat{\Sigma}_B \hat{\Sigma}_W^{-\frac{1}{2}} \varphi = \lambda \varphi$ first and scale $\varphi$ by $\hat{\Sigma}_W$, i.e. $\phi = \hat{\Sigma}_W^{-\frac{1}{2}} \varphi $.
 \end{note}

\renewcommand{\algorithmicrequire}{ \textbf{Input:}} %Use Input in the format of Algorithm  
\renewcommand{\algorithmicensure}{ \textbf{Output:}} %UseOutput in the format of Algorithm  

\begin{algorithm}[htb]   
\caption{Linear Discriminate Analysis}   
\label{alg:lda}   
\begin{algorithmic}[1] %这个1 表示每一行都显示数字  
\REQUIRE ~~\\ %算法的输入参数:Input  
 Data with label $\{X_i,y_i\}_{i=1}^N $
\ENSURE ~~\\ %算法的输出:Output  
Effective dimension reducing directions $U_k$

\STATE Compute sample mean $$\hat{\mu} = \frac{1}{N} \sum\limits_{i=1}^N X_i ;$$   
\STATE Compyte within class means $$ \hat{\mu}_k = \frac{1}{N_k} \sum\limits_{y_i=k} X_i ;$$ 
\STATE Compute Between class covariance matrix 
 $$\hat{\Sigma}_B^{p\times p} = \frac{1}{K} \sum_{k=1}^K(\hat{\mu_k} - \hat{\mu})(\hat{\mu_k} - \hat{\mu})^T ;$$  
\STATE Compute Within class covariance matrix 
 $$\hat{\Sigma}_W^{p\times p}  = \frac{1}{N-K} \sum\limits_{k=1}^K\sum\limits_{y_i=k}(X_i - \hat{\mu}_k)(X_i - \hat{\mu}_k)^T;$$
\STATE Generalized Eigen decomposition $\hat{\Sigma}_B = \hat{\Sigma}_W U\Lambda U^T $ with $\Lambda = diag(\lambda_1,\lambda_2,...\lambda_n)$ where  $\lambda_1 \ge \lambda_2 \ge ... \ge \lambda_k$ ;
\STATE Choose eigenvectors corresponding to top $k$ nonzero eigenvalues, $U_k$ i.e.
$$U_k=[u_1,\ldots,u_k], \ \ \ u_k\in \R^n ;$$ 

\RETURN $U_k$ . %算法的返回值  
\end{algorithmic}  
\end{algorithm}  



%%%%%%%%%%%%%%%%%%%%%%%%%%%%%%%%%%%%%%%%%%%%%%%%%%%%%%
\section{Sliced Inverse Regression} 


[Ker-Chau Li] \emph{Sliced inverse regression}(SIR) is a dimension reduction method using inverse regression. The idea is to find a smooth regression function that operates on a variable set of projections.
\subsection{Inverse Regression versus Regression}
\textbf{Regression:}  $f(X) = \mathop{\mathbb{E}}[y|X]$, a curve in $\mathbb{R}^{p+1}$ ;\\
\textbf{Inverse Regression:} $g(y) = \mathop{\mathbb{E}}[X|y]$, a curve in $\mathbb{R}^{p+1}$ called the inverse regression curve.
\subsection{Model}
Given a reponse variable $Y$ and a random vector $X\in \mathbb{R}^p$ of explanatory variables, SIR is based on the model 
$$Y = g(\mathnormal{\Gamma} X, \epsilon) \ ,$$
where $\mathnormal{\Gamma}^{k\times p}$ is a unknown projection and $k<p$, and $g$ is infinite dimensional with some linearity condition, not arbitrary.

\begin{algorithm}[htb]   
\caption{Sliced Inverse Regression}   
\label{alg:sir}   
\begin{algorithmic}[1] %这个1 表示每一行都显示数字  
\REQUIRE ~~\\ %算法的输入参数:Input  
 Data with label $\{X_i,y_i\}_{i=1}^N$, where $X_i\in \mathbb{R}^p$, $y_i\in \mathbb{R}$ is continuous(or ordered discrite)
\ENSURE ~~\\ %算法的输出:Output  
Effective dimension reducing directions $\mathnormal{\Gamma}_k$
\STATE Divede the range of $y_i$ into $S$ nonoverlapping slices $H_s(s = 1,...,S)$. $N_s$ is the number of observations within each slice ; 
\STATE Compute the sample mean and total covariance matrix 
  $$\hat{\mu} = \frac{1}{N} \sum\limits_{i=1}^N X_i,\quad\quad\hat{\Sigma}^{p\times p} = \frac{1}{K} \sum_{i=1}^N(X_i - \hat{\mu})(X_i -\hat{\mu})^T \ ; $$   
\STATE Compyte the mean of $X_i$ over all slices and Between slices covariance matrix 
 $$ \hat{\mu}_k = \frac{1}{N_s} \sum\limits_{y_i\in H_s } X_i, \quad \quad \hat{\Sigma}_B^{p\times p} = \frac{1}{N} \sum_{h}^K(\hat{\mu_k} - \hat{\mu})(\hat{\mu_k} - \hat{\mu})^T\  ;$$
\STATE Generalized Eigen decomposition 
$\hat{\Sigma}_B = \hat{\Sigma} U\Lambda U^T $ with $\Lambda = diag(\lambda_1,\lambda_2,...\lambda_n)$ where  $\lambda_1 \ge \lambda_2 \ge ... \ge \lambda_k$ ;
\STATE Choose generalized eigenvectors corresponding to top $k$ nonzero eigenvalues, $\mathnormal{\Gamma}_k$ i.e.
$$\mathnormal{\Gamma}_k=[u_1,\ldots,u_k], \ \ \ u_k\in \R^n ;$$ 
\RETURN $\mathnormal{\Gamma_k}$ . %算法的返回值  
\end{algorithmic}  
\end{algorithm}  
\subsection{Localized SIR}
[Wu-Liang-Mukherjee, 2009]\emph{Localized Sliced Inverse Regression}(LSIR) allows for supervised dimension reducion by projection onto a linear subspace that captures the nonlinear subspace relevant to predicting the response.

\begin{algorithm}[htb]   
\caption{Localized Sliced Inverse Regression}   
\label{alg:lsir}   
\begin{algorithmic}[1] %这个1 表示每一行都显示数字  
\REQUIRE ~~\\ %算法的输入参数:Input  
 Data with label $\{X_i,y_i\}_{i=1}^N$, where $X_i\in \mathbb{R}^p$, $y_i\in \mathbb{R}$
\ENSURE ~~\\ %算法的输出:Output  
Effective dimension reducing directions $\mathnormal{\Gamma}_k$

\STATE Compute total covariance matrix $\hat{\Sigma}$ as in SIR; 
\STATE Divede the range of $y_i$ into $S$ nonoverlapping slices $H_s(s = 1,...,S)$;
\STATE For each sample $(X_i,y_i)$ compute 
 $$ \hat{\mu}_{i,loc} = \frac{1}{|s_i|} \sum\limits_{j\in s_i } X_j ,$$
 where $s_i = \{j: x_j \text{ belongs to the $k$ nearest neighboors of $x_i$ in $H_s$}\}$ and $s$ indexes the slice $H_s$ to which $i$ belongs;

\STATE Compyte a localized version of $\hat{\Sigma}_B$ 
$$ \hat{\Sigma}_{loc}  = \frac{1}{N} \sum\limits_{i} (\hat{\mu}_{i,loc} - \hat{\mu})(\hat{\mu}_{i,loc} - \hat{\mu})^T \ ;$$
 
\STATE Solve the generalized Eigen decomposition problem 
$\hat{\Sigma}_{loc} = \hat{\Sigma} U\Lambda U^T $ with $\Lambda = diag(\lambda_1,\lambda_2,...\lambda_n)$ where  $\lambda_1 \ge \lambda_2 \ge ... \ge \lambda_k$;
\STATE Choose generalized eigenvectors corresponding to top $k$ nonzero eigenvalues, $\mathnormal{\Gamma}_k$ i.e.
$$\mathnormal{\Gamma}_k=[u_1,\ldots,u_k], \ \ \ u_k\in \R^n \;;$$ 
\RETURN $\mathnormal{\Gamma_k}$ . %算法的返回值  
\end{algorithmic}
\end{algorithm}
%%%%%%%%%%%%%%%%%%%%%%%%%%%%%%%%%%%%%%%%%%%%%%%%%%%%%%%%%
\section{Classical Multidimensional Scaling}
  \emph{Classical Multidimensional Scaling}(CMDS,MDS) aims to construct Euclidean coordinates for each object in $N$-dimensional space such that the given between-object distances are preserved as well as possible, in particular 

Given pairwise distaneces between $n$ data point 
$$ D = [d_{ij}^2] \quad where \quad d_{ii} = 0, d_{ij}\ge 0 $$

construct a system of Euclidean coordinates $\{x_i\}_{i=1}^n\in \mathbb{R}^k \quad s.t. \quad \|x_i-x_j\| =d_{ij} \quad \forall \; i,j$

\subsection{Forward Problem}

Given a set of points $x_1, x_2, ..., x_n \in \mathbb{R}^p$, let 
$$ X = [x_1, x_2, ..., x_n]^{p\times n}\;.$$
The distance between point $x_i$ and $x_j$ satisfies 
$$ d_{ij}^2 = {\|x_i - x_j\|}^2 = {(x_i-x_j)}^T(x_i-x_j) = {x_i}^T x_i + {x_j}^T x_j - 2{x_i}^Tx_j\ . $$

\subsection{Inverse Problem}
Given $d_{ij}$, find a $\{x_i\}$ satisfying the relations above. Clearly the solutions are not unique as any Euclidean transform on $\{x_i\}$ gives another solution. General ideas of classic (metric) MDS is:
\begin{enumerate}
\item transform squared distance matrix $D=[d_{ij}^2]$ to an inner product form;
\item compute the eigen-decomposition for this inner product form. 
\end{enumerate} 


\subsection{Relationship between $\tilde{X}$ and $D$}

Define inner product matrix 
$$ K_{ij} = \langle x_i,x_j\rangle, $$ 
and let $k = \diag \{K_{ii}\} \in \mathbb{R}^n,$ then 
$$D_{ij} = k_{ii} + k_{jj} - 2k_{ij} \quad or$$
$$ D = k\cdot {\mathbf{1}}^T + \mathbf{1} \cdot k^T -2K, $$
where $\mathbf{1} = (1,1,..., 1)^T \in \mathbb{R}^n$. 

For the centered data 
$$ \tilde{X}^{p\times n} = XH ,$$ 
the inner produce matrix is

\begin{eqnarray*}
\tilde{K}^{n \times n} &=& \tilde{X}^T \tilde{X}^T \\
 &=& H^T (X^TX)H \\
 &=& HKH.
\end{eqnarray*}

Construct 
\begin{eqnarray*}
B &=& -\frac{1}{2} H^T D H \\
&=& -\frac{1}{2} H^T (k\cdot {\mathbf{1}}^T + \mathbf{1} \cdot k^T -2K)H \\
&=& H^T K H \\
&=& \tilde{K},
\end{eqnarray*}
since 
$$ H^T \mathbf{1} = (I - \frac{1}{n}\mathbf{1}\mathbf{1}^T)\cdot \mathbf{1} = 0.$$

Therefore, Eign-decomposition applied to $B = -\frac{1}{2}HDH^T$ will give rise the Euclidean coordinates $\tilde{X}$. 

In practice, one often chooses top $k$ nonzero eigenvectors of $B$ for a $k$-dimensional Euclidean embedding of data.


\begin{algorithm}[htb]   
\caption{Classical MDS Algorithm}   
\label{alg:cmds}   
\begin{algorithmic}[1] %这个1 表示每一行都显示数字  
\REQUIRE ~~\\ %算法的输入参数:Input  
A squared distance matrix $ D^{n\times n} $ with $D_{ij} = d_{ij}^2$;
\ENSURE ~~\\ %算法的输出:Output  
Euclidean $k$-dimensional coordinates $\widetilde{X}_k\in \R^{k\times n}$ of data.
\STATE Compute $\displaystyle B = - \frac{1}{2} H \cdot D \cdot H^T $, where $H$ is a centering matrix;   
\STATE 
Eigenvalue decomposition\ $ B = U \Lambda U^T $ with $\Lambda=\diag(\lambda_1,\ldots,\lambda_n)$ where
$\lambda_1\geq \lambda_2\geq \ldots \geq\lambda_n\geq 0 $;\\ 
\STATE Choose top $k$ nonzero eigenvalues and corresponding eigenvectors, $ \widetilde{X}_k = U_k{\Lambda_k}^\frac{1}{2} $ where 
$$U_k=[u_1,\ldots,u_k], \ \ \ u_k\in \R^n,$$
$$\Lambda_k=\diag(\lambda_1,\ldots,\lambda_k)$$
with $\lambda_1\geq \lambda_2\geq \ldots \geq\lambda_k> 0 $;   

\RETURN $\widetilde{X}_k$; %算法的返回值  
\end{algorithmic}
\end{algorithm}

\section{Generalized MDS}

\begin{defn}[Positive Semi-definite]
Suppose $A^{n\times n}$ is a real symmetric matrix,then:\\
$A$ is \emph{positive semi-definite}($p.s.d.\; or\; A\succeq 0)$ $\iff \forall v \in \R^n,v^TAv \geq 0 \iff A=Y^TY$
\end{defn}

\begin{property}
Suppose $A^{n\times n}$, $B^{n\times n}$ are real symmetric matrix, $A\succeq 0$, $B\succeq 0$. Then we have:
\begin{enumerate}
\item $A+B\succeq 0$;
\item $A\circ B\succeq 0$;
\end{enumerate}
where $A\circ B$ is called Hadamard product and $(A\circ B)_{i,j}:=A_{i,j}\times B_{i,j}$.
\end{property}


\begin{defn}[Conditionally Negative Definite]
Let $A^{n\times n}$ be a real symmetric matrix. $A$ is \emph{conditionally negative definite}(c.n.d.) if \\
$$\forall v \in \R^n, \quad s.t.\quad \mathbf{1}^T v=\sum^n_{i=1}v_i=0,\ \text{there holds}\quad  v^TAv \leq 0$$
\end{defn}

\begin{lem}[Young/Househ\"{o}lder-Schoenberg '1938]\label{lem:1} 
For any signed probability measure $\alpha$ ($\alpha \in \R^n,\sum^n_{i=1}\alpha_i=1)$,
$$B_\alpha = -\frac{1}{2}H_\alpha C H_\alpha^T \succeq 0 \iff \mbox{$C$ is c.n.d.} $$
where $H_\alpha$ is Householder centering matrix: $H_\alpha = \mathbf{I} - \mathbf{1}\cdot \alpha^T$.
\end{lem}

Note: $u = \frac{1}{n} \cdot \mathbf{1}$ gives the perious result: squared distance matrix is c.n.d.


Sometimes, we may want to transform a square distance matrix to another square distance matrix. The following theorem tells us the form of all the transformations between squared distance matrices.

\begin{thm}[Schoenberg Transform]
Given $D$ a squared distance matrix, $C_{i,j}=\Phi(D_{i,j})$. Then
$$\mbox{$C$ is a squared distance matrix $\iff \Phi$ is a Schoenberg Transform.}$$
\end{thm}

A \emph{Schoenberg Transform} $\Phi$ is a transform from $\R^+$ to $\R^+$, which takes $d$ to 
$$\Phi (d)=\int_0^\infty \frac{1-\exp{(-\lambda d)}}{\lambda}g(\lambda)d\lambda, $$ 
where $g(\lambda)$ is some nonnegative measure on $[0,\infty)$ s.t 
$$\displaystyle \int_0^\infty \frac{g(\lambda)}{\lambda}d\lambda<\infty.$$
Examples of Schoeberg transforms include 
\begin{itemize}
\item $\phi_0(d)=d$ with $g_0(\lambda)=\delta(\lambda)$;
\item $\displaystyle \phi_1(d) = \frac{1-\exp(-ad)}{a}$ with $g_1(\lambda) = \delta(\lambda-a)$ ($a>0$); 
\item $\phi_2(d) = \ln (1 + d/a)$ with $g_2(\lambda) = \exp(-a \lambda)$; 
\item $\displaystyle \phi_3(d)=\frac{d}{a(a+d)}$ with $g_3(\lambda)=\lambda \exp(-a \lambda)$;
\item $\displaystyle \phi_4(d)=d^p$ ($p\in (0,1)$) with $\displaystyle g_4(\lambda) = \frac{p}{\Gamma(1-p)} \lambda^{-p}$. 
\end{itemize}

\section{Hilbert Space Embedding and Reproducing Kernels}
\subsection{Hilbert Space Embedding}
\begin{defn}[Positive Definite Function]
A function $k: \mathcal{X}\times \mathcal{X} \mapsto \mathbb{R} $ is a \emph{positive definite function}(p.d.f) if it is symmetric and positive definite, i.e. 
$$\sum\limits_{i,j} c_ic_jk(x_i,x_j)\ge 0, $$ 
with euqality `=' holds iff $c_n=0, \ \forall n \in \mathbb{N}$. 

Equivalently, the matrix $\{k(x_i,x_j)\}^{n\times n}$ is positive definite $\forall x_i\in \mathcal{X},\ \forall n\in \mathbb{N}$
\end{defn}
\begin{thm}[Schoenberg 38]
A separable space $M$ with a metric function $d(x,y)$ can be isometrically imbedded in a Hilbert space $H$, if and only if the family of functions $e^{-\lambda d^2}$ are positive definite for all $\lambda>0$ (in fact we just need it for a sequence of $\lambda_i$ whose accumulate point is $0$).  
\end{thm}

\subsection{Reproducing Kernels Hilbert Space} 

\begin{defn}
A Reproducing Kernel Hilbert Space (RKHS) is a Hilbert space associated with a p.d.f $k(\cdot,\cdot)$ such that:
\begin{enumerate}
  \item $k_x \in H $, where $k_x = k(x,\cdot)$ 
  \item linear span: $L:= {\sum\limits_x c_xk_x}$
  \item inner product: $\langle k_x,k_y \rangle = k(x,y)$
  extends to L: $\langle \sum\limits_x c_xk_x,\sum\limits_y c_yk_y\rangle = \sum\limits_{x,y} c_xc_yk(x,y)$
  \item closure under induced norm $\|k_x\|_H = \sqrt{k(x,x)}$
\end{enumerate}
\end{defn}
 

\begin{example} Some kernels in RKHS:
\begin{enumerate}
\item polynomial;
\item piecewise polynomial, Sobolev, Splines; 
\item wavelet(Daubeshies)
\item Mercer kernel $\mathcal{X}$ is compact, $k(x,y)\in C(\mathcal{X},\mathcal{X})$
\end{enumerate}
\end{example}

By Riesz representation, for every $x\in \X$ there exists $E_x \in \H$ such that $f(x) = \langle f, E_x\rangle$.

The evaluation functional over the Hilbert space $H$ is a linear functional that evaluates each function at a point $x$, 
$$ L_x : f \mapsto f(x) \quad \forall f\in H.$$

Evaluation functional $L_x$ is bounded, $L_x(f) = f(x) = |\langle f,E_x \rangle| \le \|f\|_H\|E_x\|_H$.

[Wahba 1990] All function models in Hilbert space in statistics are RKHS. 

\begin{thm}[Moore–Aronszajn]
Suppose K is a symmetric, positive definite kernel on a set $\mathcal{X}$. Then there is a unique Hilbert space of functions on $\mathcal{X}$ for which $k$ is a reproducing kernel, i.e., 
$$f(x) = \langle f,k(\cdot,x)\rangle _H$$ . 
\end{thm}

\begin{thm}[Mercer]
A continuous p.d.f $k$ defined on $\Omega \times \Omega $ has the expansion: 
$$k(x,y) = \sum\limits_{i = 1} ^\infty \gamma_i \phi_i(x) \phi_i(y) ,$$ 
where 
$$\int_\Omega k(x,y)\phi_i(y)\, \textit{d}\mu(y) = \gamma_i \phi_i(x) ,$$
and 
\begin{itemize}
  \item eigen values $\gamma_i \to 0 \,$;  
  \item eigen functions $\{\phi_i\}$ form an orthonormal basis of $L_2(\Omega)$. 
\end{itemize}
The converange is absolute and uniform over $\Omega\times \Omega$.
\end{thm}



\subsection{Regularization and Repeoducing Kernel Hilbert Spaces}
A general class of regularization problems has the form 
$$\min\limits_{f\in \mathcal{H}} \sum\limits_{i=1}^n \ell(y_i,f(x_i)) +\lambda \|f\|^2_{\mathcal{H}},$$
which is a infinite-dimensional problem. 

It can be shown that the solution of this problem is finite-dimensional, and has the form 
$$ f(x) = \sum\limits_{i=1}^n c_ik_{x_i}(x) .$$

\begin{example} Some loss functions \\
\textbf{Hings loss}: $\ell(y,f(x)) = \max(0, 1-yf(x))$\\
\textbf{ $\epsilon$-insensitive loss}: $\ell(y,f(x)) = \min(0, |y-f(x)|-\epsilon)$
\end{example}

\end{document}


